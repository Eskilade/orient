\documentclass{article}
\usepackage{graphicx}
\usepackage{amsmath}
\usepackage{amsfonts}
\usepackage{bm}
\usepackage[margin=1in]{geometry}
\usepackage{parskip}

\usepackage[backend=biber,style=alphabetic,sorting=ynt,bibencoding=utf8]{biblatex}
\addbibresource{sources.bib}

\renewcommand\aa{\bm{\theta}}
\renewcommand\t{\vartheta}
\newcommand\R{\bm{R}}
\newcommand\q{\bm{q}}
\newcommand\I{\bm{I}}
\renewcommand\u{\bm{u}}
\renewcommand\v{\bm{v}}
\renewcommand\skew[1]{[#1]_{\times}}
\newcommand\Real{\mathbb{R}}
\newcommand\norm[1]{\Vert#1\Vert}
\newcommand\vecop[1]{\text{vec} \left( #1\right)}

\DeclareMathOperator{\arctantwo}{arctan2}

\begin{document}

\title{Overview of 3D orientation transformations with Jacobians}
\author{Christian V. Berg}
\maketitle

\date

\begin{abstract}
The abstract text goes here.
\end{abstract}

\tableofcontents
\section{Symbols and notations}

\begin{tabular}{l p{15cm}}
  $\t \in \Real$ & Angle \\
  $\R \in SO(3)$ & Rotation matrix \\
  $\aa \in \Real^3$ & Axis-angle such that $\aa = \t \, \u$, $\t \in \Real$ and a unit vector $\u \in S^2$ \\
  $\q \in S^3$ & Unit quaternion such that $\q = \begin{bmatrix} w \\ \v \end{bmatrix}$ \\
  $\I \in \Real^{3\times3}$ & Identity matrix \\
  $\bm{0} \in \Real^{3}$ & Zero vector \\
  $ (.)^T : \Real^{m \times n} \to \Real^{n \times m}$ & Matrix transpose \\
  $tr(.) : \Real^{n \times n} \to \Real$ & Matrix trace \\
  $\skew{.} : \Real^3 \to  \Real^{3\times 3}$ & Skew symmetric matrix operator such that \newline
  $\begin{bmatrix} x \\ y \\ z \end{bmatrix}_\times =
  \begin{bmatrix}
    0& - z & y \\ 
    z & 0 & -x \\
    -y & x & 0
  \end{bmatrix}$ \\
  $(.)^\vee : \Real^{3 \times 3} \to \Real^{3}$ & Vee operator such that
  $\begin{bmatrix}
    a_1& - z & y \\ 
    z & a_2 & -x \\
    -y & x & a_3
  \end{bmatrix}^\vee = \begin{bmatrix} x \\ y \\ z \end{bmatrix}$ \newline Notice that $\left(\skew{\bm{d}}\right)^\vee = \bm{d}$ \\
    $\vecop{.} : \Real^{m \times n} \to \Real^{m\,n}$ & Vectorization operator of a matrix such that \newline
   $\vecop{\begin{bmatrix}
    a_1 & a_3 \\ 
    a_2 & a_4
    \end{bmatrix}} = \begin{bmatrix} a_1 & a_2 & a_3 & a_4 \end{bmatrix}^T$ \\ [10pt]
  $E_i \in \Real^{3 \times 3}$ & $i^{th}$ generator of $SO(3)$, such that \newline
    $E_x = 
    \begin{bmatrix}
      0 & 0 & 0 \\ 
      0 & 0 & -1\\ 
      0 & 1 & 0
    \end{bmatrix}
    E_y = 
    \begin{bmatrix}
      0 & 0 & 1 \\ 
      0 & 0 & 0\\ 
      -1& 0 & 0
    \end{bmatrix}
    E_z = 
    \begin{bmatrix}
      0 & -1& 0 \\ 
      1 & 0 & 0 \\ 
      0 & 0 & 0
    \end{bmatrix}$\\ [10pt] \\
\end{tabular}

I will use the notaion used in \cite{claraco_tutorial_nodate} for expressions
involving derivatives of matricies. For instance,
given functions $f : \Real^2 \to \Real^{2 \times 2}$ and $f_i : \Real^2 \to \Real$ such that 
\begin{align}
  f(\bm{p}) = f \left( \begin{bmatrix} x \\ y \end{bmatrix} \right) = \begin{bmatrix} f_1(x,y) & f_3(x,y) \\ f_2(x,y) & f_4(x,y) \end{bmatrix}
\end{align}
then 
\begin{align}
  \frac{\partial f(\bm{p})}{\partial \bm{p}} = \frac{\partial \vecop{f(\bm{p})}}{\partial \bm{p}} 
  &= \begin{bmatrix} 
      \frac{\partial \vecop{f(\bm{p})}}{\partial x} & 
      \frac{\partial \vecop{f(\bm{p})}}{\partial y}
    \end{bmatrix}
  = \begin{bmatrix} 
    \frac{\partial f_1(x,y)}{\partial x} & \frac{\partial f_1(x,y)}{\partial y}   \\[10pt]
      \frac{\partial f_2(x,y)}{\partial x} & \frac{\partial f_2(x,y)}{\partial y} \\[10pt]
      \frac{\partial f_3(x,y)}{\partial x} & \frac{\partial f_3(x,y)}{\partial y} \\[10pt]
      \frac{\partial f_3(x,y)}{\partial x} & \frac{\partial f_4(x,y)}{\partial y}
     \end{bmatrix}
\end{align}

Similarly, given functions $g : \Real^{2 \times 2} \to \Real^2$ and  $g_i : \Real^{2 \times 2} \to \Real$ such that 
\begin{align}
  g(M) = f \left( \begin{bmatrix} a_1 & a_3 \\ a_2 & a_4 \end{bmatrix}\right) = \begin{bmatrix} g_1(M) \\ g_2(M) \end{bmatrix}
\end{align}
then 
\begin{align}
  \frac{\partial g(M)}{\partial M} = \frac{\partial g(M)}{\partial \vecop{M}}
  &= \begin{bmatrix} 
      \frac{\partial g_1(M)}{\partial \vecop{M}} \\[5pt]
      \frac{\partial g_2(M)}{\partial \vecop{M}}
     \end{bmatrix}
  = \begin{bmatrix} 
      \frac{\partial g_1(M)}{\partial a_1} & \frac{\partial g_1(x,y)}{\partial a_2} &
      \frac{\partial g_1(M)}{\partial a_3} & \frac{\partial g_1(x,y)}{\partial a_4}\\[5pt]
      \frac{\partial g_2(M)}{\partial a_1} & \frac{\partial g_2(x,y)}{\partial a_2} &
      \frac{\partial g_2(M)}{\partial a_3} & \frac{\partial g_2(x,y)}{\partial a_4}
     \end{bmatrix}
\end{align}

Unless explicitly stated otherwise, the derivative of a matrix by a scalar remains the same as in normal matrix calculus, i.e. given functions $h : \Real \to \Real^{2 \times 2}$ and  $h_i : \Real \to \Real$ such that 
\begin{align}
  h(x) = \begin{bmatrix} h_1(x) & h_3(x) \\ h_2(x) & h_4(x) \end{bmatrix}
\end{align}
then 
\begin{align}
  \frac{\partial f(x)}{\partial x} =
    \begin{bmatrix} 
      \frac{\partial h_1(x)}{\partial x} & \frac{\partial h_3(x)}{\partial x} \\[10pt]
      \frac{\partial h_2(x)}{\partial x} & \frac{\partial h_4(x)}{\partial x}
     \end{bmatrix}
\end{align}

Similarly, and unless explicitly stated otherwise, the derivative of a scalar by a matrix remains the same as in normal matrix calculus, i.e. given a function $t : \Real^{2 \times 2} \to \Real$, then
\begin{align}
  \frac{\partial t(M)}{\partial M} =
    t \left( \begin{bmatrix} 
      a_1 & a_3 \\
      a_2 & a_4
    \end{bmatrix} \right)= 
    \begin{bmatrix} 
      \frac{\partial t(M)}{\partial a_1} & \frac{\partial t(M)}{\partial a_3} \\[10pt]
      \frac{\partial t(M)}{\partial a_2} & \frac{\partial t(M)}{\partial a_4}
     \end{bmatrix}
\end{align}

Finally, notice that 
\begin{align}
  \frac{\partial f(\bm{p})}{\partial \bm{p}}
  &= \begin{bmatrix} 
      \vecop{\frac{\partial f(\bm{p})}{\partial x}} & 
      \vecop{\frac{\partial f(\bm{p})}{\partial y}}
    \end{bmatrix}\\
  \frac{\partial g(M)}{\partial M} 
  &= \begin{bmatrix} 
      \vecop{\frac{\partial g_1(M)}{\partial M}}^T \\[5pt]
      \vecop{\frac{\partial g_2(M)}{\partial M}}^T
     \end{bmatrix}
\end{align}

\section{Rotation matrix and axis-angle}
\subsection{Rotation matrix from axis-angle}
This transformation is provided by Rodrigues' rotation formula, and gives a closed form expression of the exponential map of $SO(3)$:
\begin{align}
  \R = \exp(\skew{\aa}) = \I + \frac{\sin \t}{\t} \skew{\aa} + \frac{1 - \cos \t}{\t^2} \skew{\aa}^2 
\end{align}

First order Taylor approximation at $\aa = \bm{0}$
\begin{align}
  \R = \exp(\skew{\aa}) = \I + \skew{\aa} + O(\skew{\aa}^2) \approx \I + \skew{\aa}
\end{align}

\subsection{Rotation matrix from axis-angle Jacobian}
Since the expression is fairly long, we will define 2 intermediate functions:
\begin{align}
  k_1(\t) = \frac{\sin{\t}}{\t} \;\;&,\;\;
  \frac{\partial k_1(\t)}{\partial \t} = \frac{\t\cos{\t} - \sin{\t}}{\t^2} \\
  k_2(\t) = \frac{1-\cos{\t}}{\t^2} \;\;&,\;\;
  \frac{\partial k_2(\t)}{\partial \t} = \frac{\t\sin{\t} - 2\left(1-\cos{\t} \right)}{\t^3}
\end{align}

For both functions, we have
\begin{align}
  \frac{\partial k(\t)}{\partial \aa} &= 
    \frac{\partial k(\t)}{\partial \t} \frac{\t}{\partial \aa} =
    \frac{\partial k(\t)}{\partial \t} \frac{\sqrt{\aa^T\aa}}{\partial \aa} =
    \frac{\partial k(\t)}{\partial \t} \frac{\aa^T}{\t}
\end{align}

Using the Kronecker product $\otimes$, it then follows
\begin{align}
  \frac{\partial \vecop{k_1(\t) \skew{\aa}}}{\partial \aa}  
  &= \left( \I \otimes k_1(\t)\I \right) \frac{\partial \vecop{\skew{\aa}}}{\partial \aa}
    + \left(\skew{\aa}^T \otimes \I \right) \frac{\partial \vecop{k_1{\t}\I}}{\partial \aa} \\
  &= k_1(\t) \frac{\partial \vecop{\skew{\aa}}}{\partial \aa}
    - \left(\skew{\aa} \otimes \I \right) \vecop{\I} \frac{\partial k_1(\t)}{\partial \t} \frac{\aa^T}{\t}
\end{align}

where we used $\skew{.}^T = - \skew{.}$. Next, we also have
\begin{align}
  \frac{\partial \vecop{k_2(\t) \skew{\aa}^2}}{\partial \aa}  
  &= \left( \I \otimes k_2(\t)\I \right) \frac{\partial \vecop{\skew{\aa}^2}}{\partial \aa}
    + \left({\skew{\aa}^2}^T \otimes \I \right) \frac{\partial \vecop{k_2{\t}\I}}{\partial \aa} \\
  &= k_2(\t) \frac{\partial \vecop{\skew{\aa}^2}}{\partial \aa}
    + \left({\skew{\aa}^2} \otimes \I \right) \vecop{\I} \frac{\partial k_2(\t)}{\partial \t} \frac{\aa^T}{\t}
\end{align}

where we used ${\skew{.}^2}^T = \skew{.}^2$. Putting it all together, we get:

\begin{align}
  \frac{\partial \vecop{\R}}{\partial \aa} = 
    \frac{\partial \vecop{k_1(\t) \skew{\aa}}}{\partial \aa}  
    + \frac{\partial \vecop{k_2(\t) \skew{\aa}^2}}{\partial \aa}  
\end{align}

When $\t = 0$, we reuse the exponential map of $SO(3)$ to deduce that 
\begin{align}
  \left. \frac{\partial \vecop{\R}}{\partial \aa} \right \vert_{\aa = \bm{0}} =
  \left. \frac{\partial \vecop{\exp(\skew{\aa})}}{\partial \aa_i} \right \vert_{\aa = \bm{0}} \approx
  \left. \frac{\partial \vecop{\I + \skew{\aa}}}{\partial \aa_i} \right \vert_{\aa = \bm{0}} =
  \frac{\partial \vecop{\skew{\aa}}}{\partial \aa}
\end{align}

\subsection{Axis-angle from rotation matrix}
We begin by retrieving the angle $\t$:
\begin{align}
  tr(\R) 
    &= tr(\I) + \frac{\sin \t }{\t} tr(\skew{\aa}) + \frac{1 - \cos \t}{\t^2} tr(\skew{\aa}^2) \\
    &= 3 - 2 (1 - \cos \t) \\
  \t &=  \arccos \frac{tr(\R) - 1}{2}
\end{align}

where we used $tr(\skew{\aa}^2) =  - 2 \aa^T \aa = -2\t^2$ and $tr(\skew{.}) = 0$.

Next, and assuming $\t \neq 0 \iff tr(\R) \neq 3$ and $\t \neq \pi \iff tr(\R) \neq -1$ then
\begin{align}
  \R - \R^T 
    &= \frac{\sin{\t}}{\t} \left(\skew{\aa} - \skew{\aa}^T\right)
    + \frac{1 - \cos \t}{\t^2} \left( \skew{\aa}^2 - \skew{\aa}^{2^T} \right) \\
    &= \frac{2\, \sin \t}{\t} \skew{\aa} \\
  \aa &= \frac{\t \, \left(\R - \R^T \right)^\vee }{2\, \sin \t}
\end{align}

When $\t$ is close to $0 \iff tr(\R)$ close to $3$, we can reuse the first
order Taylor approximation from before:

\begin{align}
  \aa \approx \left( \R - \I \right)^\vee
\end{align}

When $\t = \pi \iff tr(\R) = -1$
\begin{align}
  \R = \I + \frac{2}{\t^2} \skew{\aa}^2 
  &= \I + 2\skew{\u}^2 \\
  &= \I + 2\, (\u \u^T - \I) \\
  \u \u^T = \frac{\R + \I }{2}
\end{align}

where we used $\skew{\u}^2 =  \u \u^T - \u^T \u \I$ and $\u^T \u = 1$ by definition.
Since $\u \u^T$ has the form:
\begin{align}
  \u \u^T = 
  \begin{bmatrix}
    \u_x^2 & \u_x \u_y & \u_x \u_z \\
    \u_y \u_x & \u_y^2 & \u_y \u_z \\
    \u_z \u_x & \u_z \u_y & \u_z^2
  \end{bmatrix}
\end{align}

we know that the largest element of the diagonal corresponds to the most
significant axis of rotation. Since $\t=\pi$ then the largest diagonal element
is always different from zero. Therefore, we take the square root of the largest
diagonal element, then use it to divide its corresponding column vector in
order to retrieve $\u$. Note that we cannot deduce the sign of $\u$, only the
relative "sign-ness" of each element.

\begin{align}
  \u = \pm 
    \frac{col_i \left[ \frac{\R + \I}{2} \right] }
         {\sqrt{diag_i \left[ \frac{\R + \I }{2} \right] }}
  \;,\; i = argmax \;\; diag \left( \frac{\R + \I }{2} \right)
\end{align}

So finally
\begin{align}
  \aa = \pm \pi
    \frac{col_i \left[ \frac{\R + \I}{2} \right] }
         {\sqrt{diag_i \left[ \frac{\R + \I }{2} \right] }}
  \;,\; i = argmax \;\; diag \left( \frac{\R + \I }{2} \right)
\end{align}

\subsection{Axis-angle from rotation matrix Jacobian}
Once again we will define an intermediate function and variable:
\begin{align}
  k_3(\t) = \frac{\t}{2\sin{\t}} \;\;&,\;\;
  \frac{\partial k_3(\t)}{\partial \t} = 
    \left(1 - \frac{\t}{\tan{\t}} \right) \frac{1}{2\sin{\t}} \\
  x = \frac{tr(\R) - 1}{2} \;\;&,\;\; \t = \arccos(x)
\end{align}

Consequently, we have
\begin{align}
  \frac{\partial k_3(\t)}{\partial tr(\R)} 
  &=\frac{\partial k_3(\t)}{\partial \t} 
    \frac{\partial \t}{\partial x} 
    \frac{\partial x}{\partial tr(\R)} \\
  &=-\frac{\partial k_3(\t)}{\partial \t} 
    \frac{1}{2\sqrt{1-\t^2}}
\end{align}

Assuming $\t \neq 0 \iff tr(\R) \neq 3$ and $\t \neq \pi \iff tr(\R) \neq -1$, then
\begin{align}
  \frac{\partial \vecop{\aa}}{\partial \vecop{\R}} = 
    \big({(\R - \R^T)^\vee}^T \otimes \I \big)
    \vecop{\I}
    \frac{\partial k_3(\t)}{\partial tr(\R)} 
    \frac{\partial tr(\R)}{\partial \vecop{\R}} + 
    k_3(\t) \frac{\partial \left( \R - \R^T \right)^\vee}{\partial \vecop{\R}}
\end{align}

When $\t = 0 \iff tr(\R) = 3$, then 
\begin{align}
  \left. \frac{\partial \aa}{\partial \vecop{\R}} \right \vert_{\aa = \bm{0}} \approx
    \frac{\partial \left(\R - \I \right)^\vee}{\partial \vecop{\R}}
\end{align}

\section{Rotation matrix and quaternion}
\subsection{Rotation matrix from quaternion}
\begin{align}
  \R = \I + 2 \; w \; \skew{\v} + 2\; \skew{\v}^2
\end{align}

\subsection{Rotation matrix from quaternion Jacobian}
\begin{align}
  \frac{\partial \vecop{\R}}{\partial w} &= 2\, \vecop{\skew{\v}} \\
  \frac{\partial \vecop{\R}}{\partial \v} &= 
  2w \frac{\partial\vecop{\skew{\v}}}{\partial \v} +
  2 \frac{\partial\vecop{\skew{\v}^2}}{\partial \v} \\
  \frac{\partial \vecop{\R}}{\partial \q} &= 
    \begin{bmatrix}
      \frac{\partial \vecop{\R}}{\partial w} & 
      \frac{\partial \vecop{\R}}{\partial \v}
    \end{bmatrix}
\end{align}

\subsection{Quaternion from rotation matrix}
We begin by retrieving the scale component of the quaternion:
\begin{align}
  tr(\R) 
    &= tr(\I) + 2\; w \; tr(\skew{\v}) + 2 \; tr(\skew{\v}^2) \\
    &= 3 - 4 \v^t \v = 4\;w^2 - 1 \\
  w &= \pm \frac{ \sqrt{ tr(\R) + 1} }{2}
\end{align}

Assuming $w \neq 0 \iff tr(\R) \neq -1$
\begin{align}
  \R - \R^T
    &= 2\; w \; \left( \skew{\v} - \skew{\v}^T \right) + 2 \; \left(\skew{\v}^2 - {\skew{\v}^2}^T \right) \\
    &= 4\; w \; \skew{\v} \\
  \v &= \frac{\left( \R -\R^T \right)^\vee}{4\;w}
\end{align}

When $w = 0$, then
\begin{align}
  \R &= \I + 2 \skew{\v}^2  = \I + 2 \left ( \v \v^T - \I \right) \\
  \v \v^T &= \frac{\R + \I}{2}
\end{align}

and we can reuse the same method as for axis-angle when $\t = \pi$.

\subsection{Quaternion from rotation matrix Jacobian}
As before, we define
\begin{align}
  k_4(w) = \frac{1}{4w} \;\;,\;\; \frac{\partial k_4(w)}{\partial w} = -\frac{1}{4w^2}
\end{align}

Assuming $w \neq 0 \iff tr(\R) \neq -1$
\begin{align}
  \frac{\partial w}{\partial \vecop{\R}}
    &=\frac{\partial w}{\partial tr(\R)} \frac{\partial tr(\R)}{\partial \vecop{\R}}
    = \frac{1}{4\sqrt{tr(\R) + 1}} \frac{\partial tr(\R)}{\partial \R} \\
  \frac{\partial \v}{\partial \vecop{\R}}
    &= \big({\left(\R - \R^T\right)^\vee}^T \otimes \I \big)\frac{\partial k_4(w)}{\partial w}\frac{\partial w}{\partial \vecop{\R}}
    + k_4(w) \frac{\partial \left(\R - \R^T\right)^\vee }{\partial \vecop{\R}}
\end{align}

\section{Axis-angle and quaternion}
\subsection{Quaternion from axis-angle}
\begin{align}
  \q = 
    \begin{bmatrix} 
       \cos{ \frac{\t}{2} } \\[5pt]
       \frac{\sin{ \frac{\t}{2} }}{\t} \aa
    \end{bmatrix}
\end{align}

When $\t$ is close to $0$, then the second order Taylor approximation is
\begin{align}
  \q \approx
    \begin{bmatrix} 
      1 - \frac{\t^2}{8} \\[5pt]
      \left( \frac{1}{2} - \frac{\t^2}{48} \right) \aa
    \end{bmatrix}
\end{align}

\subsection{Quaternion from axis-angle Jacobian}
\begin{align}
  \frac{\partial w}{\partial \aa} 
  &=  \frac{\partial w}{\partial \t} \frac{\partial \t}{\partial \aa}  \\
  &=-\frac{\sin{\frac{\t}{2}}}{2} \frac{\aa^T}{\t}  \\
     \frac{\partial \v}{\partial \aa} &= 
  \aa \frac{\partial \frac{\sin{ \frac{\t}{2} }}{\t}}{\partial \aa} + 
  \frac{\sin{ \frac{\t}{2} }}{\t} \frac{\partial \aa}{\partial \aa}\\
  &= \aa \frac{\partial \frac{\sin{ \frac{\t}{2} }}{\t}}{\partial \t} \frac{\partial \t}{\partial \aa}  + 
  \frac{\sin{ \frac{\t}{2} }}{\t} \I \\
  &= 
     \aa \frac{\t \, \cos{\frac{\t}{2}} - 2\,\sin{\frac{\t}{2}}}{2\,\t^2} \frac{\aa^T}{\t}
    + \frac{\sin{\frac{\t}{2}}}{\t} \I
\end{align}

When $\t$ is close to $0$, then
\begin{align}
  \left. \frac{\partial w}{\partial \aa} \right \vert_{\aa = 0} &\approx - \frac{1}{4} \aa^T \\
  \left. \frac{\partial w}{\partial \aa} \right \vert_{\aa = 0} &\approx - \frac{1}{24} \aa \aa^T + \left(\frac{1}{2} - \frac{\t^2}{48} \right) \I
\end{align}

and finally

\begin{align}
  \frac{\partial \q}{\partial \aa} = 
    \begin{bmatrix}
      \frac{\partial w}{\partial \aa} \\[5pt]
      \frac{\partial \v}{\partial \aa}
    \end{bmatrix}
\end{align}

\subsection{Axis-angle from quaternion}
Using $y=\sqrt{\v^T\v}$
\begin{align}
  \aa = 2 \frac{\arctantwo \left(y, w \right)}{y} \v 
\end{align}

When $w$ approach $1$ from the left $\iff \t$ close to $0$, we can reuse the second order Taylor approximation from before to deduce
\begin{align}
  \aa \approx \frac{6\v}{w+2}
\end{align}

\subsection{Axis-angle from quaternion Jacobian}
Once again we define an intermediate function
\begin{align}
  k_5(y,x) &= 2 \frac{\arctantwo \left(y, w \right)}{y} \\
  \frac{\partial k_5(y,x)}{\partial x} &= \frac{-2}{y^2 + w^2} = -2 \\
  \frac{\partial k_5(y,x)}{\partial y} &= 2 \frac{ \frac{w\,y}{y^2 + w^2} - \arctantwo \left(y, w \right)}{y^2} = 2 \frac{ w\,y - \arctantwo \left(y, w \right)}{y^2}
\end{align}
where we used $y^2 + w^2 = \q^T \q = 1$ since we are assuming unit quaternions. It then follows:
\begin{align}
  \frac{\partial \aa}{\partial w} &= \frac{\partial k_5(y,w)}{\partial w} \v \\
  \frac{\partial \aa}{\partial \v} 
  &= \v \frac{\partial k_5(y,w)}{\partial y} \frac{\partial y}{\partial \v} + k_5(y,w) \I
  =  \v \frac{\partial k_5(y,w)}{\partial y} \frac{\v^T}{y} + k_5(y,w) \I
\end{align}

When $w$ approach $1$ from the left $\iff \t$ close to $0$, then
\begin{align}
  \left. \frac{\partial \aa}{\partial w} \right \vert_{w = 1^-} &\approx - \frac{6}{\left( w + 2 \right)^2} \v \\
  \left. \frac{\partial \aa}{\partial \v} \right \vert_{w = 1^-} &\approx \frac{6}{w + 2} \I
\end{align}

and finally

\begin{align}
  \frac{\partial \aa}{\partial \q} = 
    \begin{bmatrix}
      \frac{\partial \aa}{\partial w} &
      \frac{\partial \aa}{\partial \v}
    \end{bmatrix}
\end{align}


\printbibliography

\end{document}
